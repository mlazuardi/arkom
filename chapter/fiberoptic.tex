%Fiber Optik (Arsitektur Komputer)
%Kelas : D4 TI 1B
%Khadijah Hasanah Putri Harahap 1174022
%Liyana Majdah Rahma 1174039
%Luthfi Muhammad Nabil 1174035
%Nisrina Aulia Firdaus 1174098
%Salwaa Tania 1174047
%Septia Rahayu 1174044
%Diana Satima Gistivani 1154018

\section{Fiber Optic}
Serat optik (fiber optic) merupakan suatu pemandu gelombang cahaya (light wave guide) yang berupa suatu kabel tembus pandang (transparant), 
yang mana pemampang dari kabel yang disebut “Cladding”, Cladding pada serat optik (fiber optic) membungkus atau mengelilingi Core. 
Adapun bentuk pemampang dari core dapat bermacam-macam antara lain: pipih, segi tiga, segi empat, segi banyak atau berbentuk lingkaran. 

\subsection 
Teknologi serat optik (fiber optic) ini akan memberikan kemungkinan yang lebih baik bagi jaringan telekomunikasi, terutama dalam hal telekomunikasi data. 
Serat optik (fiber optic) adalah salah satu media transmisi yang dapat menyalurkan informasi dengan kapasitas besar dengan tingkat keandalan (performance) yang tinggi. 
Beda dengan media transmisi lain, pada teknologi serat optik (fiber optic) ini, gelombang pembawanya tak lagi merupakan gelombang elektromagnetik (microwave) atau listrik, 
tapi merupakan sinar atau cahaya laser. 
Kabel serat optik (fiber optic) mampu melayani transfer data dengan kecepatan tinggi dalam waktu yang relatif singkat dan bentuk fisi yang relatif kecil dan ringan.
