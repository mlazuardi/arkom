\cite{gao1998introduction} Standar V.90 yang telah diratifikasi mendefinisikan karakteristik utama modem 56K sebagai berikut:
• Mode operasi dupleks melalui jaringan telepon tetap (PSTN) dan jaringan digital yang diaktifkan.
Penggunaan teknik pembatalan gema untuk pemisahan saluran.
Modulasi PCM ke hilir pada tingkat simbol 8 k dan modulasi V.34 hulu.
• Tingkat sinyal data kanal sinkron turun dari 28 Kbps menjadi 56 Kbps dengan penambahan 1,3 Kbps dan hulu dari 4,8 Kbps menjadi 33,6 Kbps dengan penambahan 2,4 Kbps.
• Modem menggunakan teknik adaptif untuk mencapai sedekat mungkin dengan tingkat sinyal data maksimum yang didukung oleh saluran pada setiap koneksi.
• Jika sambungan tidak mendukung V.90, modem jatuh kembali ke operasi V.34 dupleks penuh.
Selama dimulainya modem, laju sinyal data ditetapkan dengan urutan nilai tukar.
• Prosedur automode V.32bis dan mesin faksimili Grup 3 mendukung modem Automoding ke V.Series.
• V.8 dan secara opsional, prosedur V.8bis tersedia saat start up modem atau seleksi.
