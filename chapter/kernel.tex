%Nama Kelompok: Kernel
%Kelas: D4 1B
%Anfhaz 1174048
%Dika 1174050
%Teddy 1174038
%Surya 1174036
%Ikhas al azis 1174049
%Rangga 1174056
%Syahriyan 1174037

\section{Kernel}
 Kernel adalah program komputer yang merupakan inti dari sistem operasi komputer, dengan kontrol penuh atas segala hal yang ada di sistem. Pada kebanyakan sistem, ini adalah salah satu program pertama yang dimuat saat start-up (setelah bootloader). 
 Ini menangani sisa start-up serta permintaan input / output dari perangkat lunak, menerjemahkannya ke dalam instruksi pengolahan data untuk unit pemrosesan pusat. Ini menangani memori dan periferal seperti keyboard, monitor, printer, dan speaker.
 Kernel menghubungkan perangkat lunak aplikasi ke perangkat keras komputer.
 Kode kritis kernel biasanya dimuat ke dalam area lindung memori, yang mencegahnya ditimpa oleh aplikasi atau komponen lain yang lebih kecil dari sistem operasi. Kernel menjalankan tugasnya, seperti menjalankan proses dan penanganan interupsi, di dalam ruang kernel. 
 Sebaliknya, semua yang dilakukan pengguna ada di ruang pengguna: menulis teks di editor teks, menjalankan program di GUI, dll. Pemisahan ini mencegah data pengguna dan data kernel tidak saling mengganggu dan menyebabkan ketidakstabilan dan kelambatan. 
 Antarmuka kernel adalah lapisan abstraksi tingkat rendah. Ketika sebuah proses membuat permintaan dari kernel, itu disebut system call. Desain kernel berbeda dalam cara mereka mengatur panggilan dan sumber sistem ini. 
 Kernel monolitik menjalankan semua instruksi sistem operasi di ruang alamat yang sama untuk kecepatan. Sebuah mikrokernel menjalankan sebagian besar proses di ruang pengguna, untuk modularitas. 

\subsection{Sejarah Kernel}
Kernel merupakan program komputer yang mengatur semua permintaan akan input/output dari perangkat lunak atau software.
Pada tahun 1990an, sebuah jenis algoritma pembelajaran baru dikembangkan, berdasarkan hasil teori pembelajaran statistik: Support Vector Machine (SVM). 
Hal ini memunculkan kelas baru secara teoritis SVM - kernel - untuk sejumlah tugas pembelajaran. 
Mesin kernel menyediakan kerangka kerja modular yang dapat disesuaikan dengan berbagai tugas dan domain dengan pilihan fungsi kernel dan algoritma dasar. 
Mereka mengganti jaringan syaraf tiruan di berbagai bidang, termasuk teknik, pencarian informasi, dan bioinformatika.
Belajar dengan Kernel memberikan pengenalan SVM dan metode kernel terkait. Meski buku ini diawali dengan dasar-dasar, namun juga mencakup penelitian terbaru. 
Ini menyediakan semua konsep yang diperlukan untuk memungkinkan pembaca menggunakan algoritma yang hebat yang telah dikembangkan melalui algoritma kernel dan untuk memahami dan menerapkan algoritma hebat yang telah dikembangkan selama beberapa tahun terakhir.
Sejarah Linux dimulai pada tahun 1991, ketika mahasiswa Universitas Helsinki, Finlandia bernama Linus Benedict Torvalds menulis Linux, sebuah Kernel untuk proses 80386
proses 32-bit pertama dalam kumpulan CPU intel yang cocok untuk PC.
Pada awal perkembangannya, sourche code Linux di sediakan secara bebas melalui internet. Kernel Linux berbeda dengan sistem Linux. Kernel Linux merupakan sebuah perangkat lunak
orisinil yang dibuat oleh komunitas Linux, sedangkan sistem Linux, yang dikenal saat ini, mengandung banyak komponen yang dibuat sendiri atau dipinjam dari berbagai proyek pengembangan lain.
Kernel Linux pertama kali yang dipublikasikan adalah versi 0.01, pada tanggal 14 Maret 1991. Sistem berkas yang didukung hanya sistem berkas Minix. Kernel pertama dibuat berdasarkan
kerangka Minix (sistem UNIX kecil yang dikembangkan oleh Andy Tanenbaum). Tetapi, kernel tersebut sudah mengimplementasi proses UNIX secara tepat.
Pada tanggal 14 Maret 1994 dirilis versi 1.0, yang merupakan tonggak sejarah Linux. Versi ini adalah kulminasi dari tiga tahun perkembangan yang cepat dari kernel Linux. Fitur baru terbesar
yang disediakan adalah jaringan. Versi 1.0 mampu mendukung protokol standar jaringan TCP/IP. Kernel 1.0 juga memiliki sistem berkas yang lebih baik tanpa batasan-batasan sistem berkas Minix.
Sejumlah dukungan perangkat keras ekstra juga dimasukkan ke dalam rilis ini. Dukungan perangkat keras telah berkembang termasuk diantaranya floppy-disk, CD-ROM, sound card, berbagai mouse,
dan keyboard internasional. Dukungan juga diberikan terhadap modul kernel yang dynamically loadable dan unloadable.
Satu tahun setelah versi 1.0 dirilis, kernel 1.2 keluar. Kernel versi 1.2 ini mendukung variasi perangkat keras yang lebih luas. Pengembangan telah memperbarui networking stack untuk menyediakan
support bagi protokol IPX, dan membuat implementasi IP lebih lengkap dengan memberikan fungsi accounting dan firewalling. Kernel 1.2 ini merupakan kernel Linux terakhir yang hanya bisa di PC. 

\subsection{Versi Kernel}
\subsubsection{Monolithic}
 Kernel Moonolithic memiliki seluruh servis dasar dari sistem operasi didalamnya. Kelebihan dari disain Monolithic adalah Efesiensi, sehingga performa sistem juga
 meningkat. Monolithic juga memiliki kelemahan, salah satunya dalam hal stabilitas, dimana kemungkinan sistem crash lebih besar. Monolithic Kernel meliputi semua 
 fungsi Kernel di satu modul. Monolithic kernel meliputi semua fungsi kernel di satu modul. Aplikasi dapat memanfaatkan fungsi kernel melalui sistem pemanggil. Alamat untuk kernel terpisah dari aplikasi untuk melindungi dari kekeliruan operasi aplikasi. Kernel menjadi sangat besar karena menyediakan beberapa fungsi untuk memuaskan permintaan user dan sekarang masalah mulai bermunculan;
 - Lemahnya Fleksibelitas
 - Modifikasi dari kernel memberi rekonfigurasi dan rekompilasi dari kernel dan pengulangan. Rekonfigurasi dan rekompilasi dari kernel memakan banyak waktu, dan operasi pengulangan tidak diinginkan untuk sistem non-stop.  
 - Portabilitas Rendah
 - Masuknya beberapa fungsi permintaan dan perbaikan pemanfaatan sistem, kode dari kernel menjadi sangat komplek.
 - Menyianyiakan Bar Alamat
 - Monolithic kernel termasuk beberapa fungsi dan sebagian dari mereka keluar dari penggunaan atau crash di beberapa aplikasi. Fungsi ini menyia-nyiakan bar alamat.
	
\subsubsection{Microkernel}
 Disain Micorkernel hanya mengimplemetasikan servis dasar minimal yang diperlukan, yaitu manajemen pengalaman memori, manajemen proses, dab inter-proses communication.
 Kelebihan Micokernel adalah stabilitas sistem lebih terjaga dan kekurangannya adalah komunikasi antara proses menjadi lebih rumit sehingga sistem menjadi tidak efisien.
 Ide MincroKernel ditemukan dengan antusias di dalam organisasi penelitian untuk membangun post-Unix sistem operasi, Hardware baru, aplikasi baru, dan metode pemograman baru
 menurut konsep sistem operasi. Ide micro-kernel ditemukan  dengan antusias di dalam organisasi penelitian untuk membangun post-Unix sistem operasi, hardware baru, aplikasi baru,
 dan metode pemprogaman baru menurut konsep sistem operasi. Sesuai objek dan mekanisme,tempat alamat, prosedur remot pemanggil, sms dasar IPC, dan kelompok komunikasi, lebih dasar dan lebih general dari microkernels industri.
	
\subsubsection{Hybrid Kernel}
 Disain Hybrid Kernel menyerupai Micokernel tetepi dengan tambahan kode yang menyebabkan Hybird Kernel dapat berjalan lebih cepat dari Micokernel. 
 Di PAF Kernel, fitur tempat dianggap sebagai bagian intergal yang termasuk dalam predikat dan salah satu dari argumen nya. Kami mencatat dan yang lain disebut fitur 
 Constituent Structure. Dua fitur ini memberikan informasi yang berbeda. Fitur Path mendeskripsikan informasi antara sebuah predikat dan argumen itu sementara fitur 
 Constituent Structure menyimpan informasi tentang struktur syntax.
	
\subsubsection{ExoKernel}
 Disain ExoKernel masih merupakan disain eksperimental dan dalam tahap penelitian sehingga belum dipakai secara luas. Perbedaan konsep disain ExoKernel dengan disain Kernel 
 lainnya adalah ExoKernel memiliki fungsi perlindungan dan pembagian resource untuk hardware. Kelebihan ExoKernel adalah bisa dimasukkan libary sistem operasi lebih dari satu 
 sehingga bisa menjalankan program-program untuk sistem operasi yang berbeda secara bersamaan. 
	
\subsubsection{Windows Kernel}
 Akar Windows mencapai kembali ke akhir 1980-an. Kembali
 Kemudian, banyak hal menarik terjadi di op-
 Ruang desain sistem erating - termasuk SVR4, Mach
 microkernel, inovasi dalam networking dan windowing sys-
 tems, dan banyak proyek penelitian berbasis OS. Itu
 keinginan untuk mendapatkan pengetahuan mendalam tentang pengembangan yang menarik ini-
 ops memotivasi banyak siswa CS untuk belajar operasi
 sistem saat itu. Dengan proyek OS kami, kami ingin membantu
 Minat kembali minat pada sistem operasi lagi.
 Dalam makalah ini, kami menganjurkan pendekatan langsung terhadap-
 lingkungan pengajaran (dan pembelajaran) konsep OS. Kami menyajikan kami
 pengalaman dari pengajaran program OS berbasis Windows dur-
 dalam sepuluh tahun terakhir ini. Kami menyarankan skema tiga fasa,
 dimana siswa pertama belajar menguasai
 kamu
 sistem ser-mode di-
 Koraces (U) - sering disebut sebagai "pemrograman sistem".
 Kedua, mereka perlu menguasai prinsip dan alat untuk mon-
 itor dan perilaku OS easure (M). Dan ketiga, siswa
 harus disajikan dengan rincian pelaksanaan utama OS
 kernel (K). Mengikuti Pendekatan UMK , bahkan com-
 proyek yang rumit seperti modifikasi pelaksanaan
 manajemen memori di dalam kernel Windows bisa jadi mobil-
 mengikuti kurikulum OS sarjana. Undertakings,
 seperti proyek Manajemen Memori Abstrak (AMM)
 mengintegrasikan dengan baik dengan courseware kami yang telah dikembangkan sebelumnya -
 Kit Sumber Daya Kurikulum Microsoft Windows Internals(CRK).
 Microsoft membuat source kernel Windows secara luas memanfaatkan-
 mampu akademisi di tahun 2006 , menggantikan yang sebelumnya terbatas
 distribusi yang tersedia hanya untuk memilih universitas.
 Sejak itu, kami telah memperluas penggunaan Windows sebelumnya
 dalam kursus OS dengan mengembangkan sejumlah proyek dan laboratorium
 yang mengandalkan modifikasi kernel Windows. Proyek ini
 fokus pada topik seperti penjadwalan / pengiriman, sinkronisasi-
 dan pengelolaan memori. Dalam tulisan ini, kita
 Hadirkan Manajemen Memori Abstrak (AMM)
 yang terdiri dari bagian U, di mana siswa prac-
 API sistem yang relevan (seperti fungsi Windows API
 VirtualAllocEx, bagian M, dimana kita bertanya kepada siswa
 untuk membiasakan diri dengan teknik pengukuran dan
 alat (seperti monitor kinerja Windows - perf-mon.exe), dan bagian K	dimana siswa perlu memodifikasi
 kode sumber (mis., ntos / mm / wsmanage.c), kompilasi, dan jalankan
 versi Windows mereka sendiri. Selama kursus, proyek
 ditugaskan ke kelompok tiga siswa.
 Dalam sisa makalah ini, pertama-tama kami menyajikan ikhtisar 490
 tentang proyek yang kami buat untuk WRK. Lalu, kami hadir
 bagian kernel (K) dan pengukuran (M) dari AMM
 proyek. (Kami telah menghilangkan bagian mode pengguna (U) karena
 keterbatasan ruang). Sebaliknya, kami menyajikan umpan balik dari stu-
 penyok yang mengambil kursus kami Akhirnya, kita menyimpulkan makalahnya
 dengan prospek proyek UMK masa depan.
	
\subsubsubsection{Duplikasi Windows Kernel}
 Untuk mencegah aplikasinya
 ion untuk menyimpan duplikat dari
 konten yang dilindungi, Windows Kernel Hook digunakan untuk mengubah
 perilaku "Save" oleh modi
 memamerkan fungsi yang sesuai
 alamat. Akibatnya, aplikasi tidak bisa menyelesaikan ini
 operasi berhasil dan tidak duplicate benar-benar diselamatkan.
 Melalui penelitian, kami menentukan
 e fungsi kernel kunci masuk	
 Proses menabung duplikat, yaitu "ZwWriteFile" yang mana bertanggung jawab untuk mengoperasikan tugas menulis. Dengan memuat NT Sopir, kita bisa menimpa alamat ZwWriteFile fungsi di SSDT dengan alamat fungsi kait
 NewZwWriteFile). Dalam keadaan seperti ini, NewZwWriteFile akan dipanggil kapan sistem bermaksud untuk memanggil ZwWriteFile. Di NewZwWriteFile , kita bisa memanggil fungsi aslinya ZwWriteFile
 dengan dimodifikasi parameter dan run re nya sults akan dikembalikan ke NewZwWriteFile, sehingga yang terakhir bisa menutupi kegagalan panggilan.
	
\subsection{Kernel Linux}
 Kernel Linux adalah salah satu proyek open-source yang paling menarik namun paling tidak dipahami. Ini juga merupakan dasar untuk mengembangkan kode kernel baru. 
 Itulah sebabnya Sams sangat antusias untuk membawa Anda informasi pengembangan kernel Linux terbaru dari orang dalam Novell di edisi kedua Pengembangan Kernel Linux. 
 Panduan praktis dan otoritatif ini akan membantu Anda lebih memahami kernel Linux melalui cakupan terkini dari semua subsistem utama, fitur baru yang terkait dengan kernel Linux 2.6 dan informasi orang dalam mengenai perkembangan yang belum pernah dirilis. 
 Anda dapat melihat kernel Linux secara mendalam dari sudut pandang teoritis dan penerapan saat Anda membahas berbagai topik, termasuk algoritme, antarmuka panggilan sistem, strategi paging dan sinkronisasi kernel. 
 Dapatkan informasi terbaik dari sumber di Linux Kernel Development.
	
\subsection{Kernel Android}
 Pertama-tama, kernel Linux perlu dikompilasi
 sesuai dengan perangkat kerasnya. File konfigurasi
 (file defconfig / .config) harus dimodifikasi agar sesuai dengan teknis
 spesifikasi perangkat keras Spesifikasi perangkat keras
 perangkat keras dapat ditentukan dengan menggunakan alat yang tersedia
 jaring (misalnya Database WURFL, yang merupakan singkatan dari Wireless
 File Sumber Universal).
 Ini memastikan bahwa versi kernel tertentu akan
 jalankan pada hardware dan support File System yang ada
 telah dibangun untuk perangkat keras.
 Setelah kita memiliki file konfigurasi yang benar, kernel perlu 
 ditambal untuk mendukung perangkat keras. Jika kernelnya adalah
 dari pohon kernel Linux, perlu ditambal untuk mendukungnya
 Android juga. Jika kernelnya adalah kernel Android, tambalan hanya untuk
 mendukung Platform perlu diterapkan.
 Patch membuat kernel yang kompatibel dengan Android dan
 platform