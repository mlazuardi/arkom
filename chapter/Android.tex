% Nama Kelompok : Android OS
% Kelas : 1A
% 1. Daffa Naufali		-
% 2. Muhammad Dzihan	-        
% 3. Nurrezky Asman		- 1174019
% 4. Yusuf Al-Qardhawi 	- 1174085

\section(Pengertian dan Sejarah Android)
	Android merupakan Program Operating System yang di buat dengan UNIX Based dan bawaan Sistem Kernel
	pada Bagian Hardware. Android \ref{androidfigures} pun di rilis tahun 2009 menggunakan bahasa pemrograman Java saat peluncuran pertamanya yang
	di sebarkan pada lingkungan masyarakat berdasarkan \cite(rasjid2015android). Ketika teknologi semakin maju berkembang, Android ini memberikan dampak baik yang sangat positif
	yang menjadikan Android tersebut semakin terkenal pada semua orang sesuai platform yang semakin fleksibel untuk dipakai.

\subsection(Fitur yang diluncurkan pada Android)
	Android telah menyelesaikan perkembangan dalam kurung waktu panjang ketika menghadirkan Aplikasi berguna untuk di gunakan dengan gratis berasal dari Sistem Android . Di awali
	dengan Multimedia, Games, Mode Penelitian, dan lain-lain. Fitur-Fitur tersebut memiliki kelebihan positif yang memberikan dampak pada Era Masa Depan.
	Waktu yang secara Real-Time ini membuat semakin mempercepat pengguna Android untuk saling komunikasi sesama yang lain. Karena Fitur tersebut
	membuat kita dapat melakukan Percakapan di mana saja dengan adanya koneksi internet dan Wifi untuk memudahkan sosialisasi ke masyarakat.
	Tidak hanya itu saja, Platfrom OS Android sudah dihadirkan pada pengguna ponsel atau smartphone yang memiliki fitur lebih.
	Dari Segi penampilan yang hampir sama dengan Mac OS dimana kumpulan icon tercantum di tengah bawah. Dan Tampilan yang elegan dan mudah
	dipandang keindahannnya.

\subsubsection(Fitur-Fitur tersembunyi di Android)
	Banyak sekali fitur-fitur yang disembunyikan oleh pengembang android yaitu sebagai berikut :
	1.	Pada bagian phone call ketika anda menginput *#*#4636#*#* ini akan langsung memunculkan sebuah settingan
		smartphone yang bisa mengubah pilihan anda yang terlock jadi terlihat menyeluruh. Pengaturan ini termasuk
		mode pengembang karena untuk sebagai fitur tambahan apabila terjadi permasalahan pada Smartphone Android
		tersebut.
	2.	Fitur Mode Akses Pengembangan yang tersembunyi pada bagian pengaturan android yang sengaja ditutup bertujuan menghindari terjadinya pemalsuan
		data di android anda. Hanya saat ini pengembang android membuat sebuah jendela pintas untuk mengakses fitur ini dengan cara membuka pengaturan
		,lalu pilih tentang ponsel anda dan pilih versi Baseband dengan menekan berkali-kali lalu akan muncul pemberitahuan di layar anda
	3.	Fitur Droidboot Android yang ada di beberapa smartphone tertentu seperti ASUS untuk saat ini. Fungsi dari Droidboot ini sebagai tempat boot os android yang mengalami stuck
		pada bagian booting up smartphone tersebut. Bukan itu saja, tetapi fitur ini bisa menginstal dan mengupgrade versi android anda ke versi baru apabila kompatibel terhadap
		versi smartphone anda saat ini. Tidak semua Smartphone Android OS di upgrade ke versi terbaru. Karena perbedaan masalah driver pada smartphone anda. Tapi anda bisa download manual
		di system updater di OS Android untuk mencari versi terbaru OS Android ini.
	4.	Fitur Pembersihan Memori atau Cleaner Memory yang digunakan untuk mengoptimasikan performa Android anda sebagai fitur keseimbangan operasi sistem pada smartphone android
	5.	Fitur Battery Saver atau Hemat Daya berguna untuk meniminimalisirkan penggunaan baterai dan aplikasi sebagai penghematan daya pakai smartphone
		sehingga bisa lebih bertahan lama.
	6.	Fitur Mode Pesawat atau Airplane Mode. Mungkin sebagian orang menggangap hal tersebut aplikasi yang sepele, Tetapi Fitur ini sangat berguna saat anda berpergian pada segi keamanan
		Fitur ini di gunakan sebagai Jaringan Tertutup untuk mematikan seluruh jaringan yang masuk pada smartphone. Manfaat Mode Pesawat ini saat anda dalam berada pesawat memiliki keselamatan
		kerja yang sangat tinggi untuk menghindari gangguan sinyal antara sinyal penyedia layanan dan sinyal bandara. Selain Manfaat tersebut, Mode Pesawat ini juga sebagai fitur pengganti Hemat Daya yang relatif lama
		karena semua jaringan yang masuk ditutup pada pengaturan ini untuk tidak menerima jaringan apapun dan dijadikan tempat tertutup menerima sinyal dari berbagai wilayah.
	7. 	Fitur Auto-System. Sistem otomatis ini sebagai pengaturan smartphone saat di biarkan pada waktu tertentu. Mereka akan menjalankan tugas yang sudah diatur oleh penggunanya agar mereka harus berhenti dipakai pada waktu tertentu
		dan kapan saat mereka digunakan lagi. Fitur tersebut adalah pengaturan waktu android kapan itu berhenti dan berjalan kembali.
	8.	Fitur Quick Boot atau Nyala Cepat. Fitur ini mempercepat langkah untuk menyalakan smartphone anda saat kondisi mengalami permasalahan saat menunggu OS Android berjalan sebagaimana mestinya.
		Fitur yang diberikan hanya di gunakan saat-saat tertentu apabila OS Android ini sangat berkerja lambat saat memunculkan Aplikasi Android di Smartphone anda.
	9.	Fitur Versi Android yang dirahasiakan oleh pembuat android. Fitur versi android ini mempunyai keunikan tersendiri. Jika anda menekan versi android berkali-kali akan memunculkan User Interface
		dengan tampilan versi android anda saat ini. Biasanya User Interface ini mempunyai aplikasi tersembunyi, mulai dari tampilan yang berisi permainan sederhana dan tampilan ikon versi android yang bergerak sendiri.
		sama halnya dengan screen saver di OS Windows.
	10.	Fitur USB Debugging atau Pengamanan USB. Mungkin ini sebagian besar fitur ini untuk mencegah terjadinya over handle aplikasi yang berjalan saat anda menggunakan smartphone android sambil mengisi daya baterai
		tersebut. Selain itu Fitur ini membuat layar smartphone anda akan menyala terus untuk melihat perkembangan daya isi baterai dan status sistem anda sekarang ini.
	
	\section(Versi-Versi Platform Android)
		Versi Android ini sendiri banyak sekali yang harus diperbaiki untuk pertama kali peluncurannya pada tahun 2009. Android ini belum memberikan sebuah nama OS Platform
		saat penyebaran berlangsung. Seiring banyak penelitian pengembangan android muncul versi-versi berikut ini: Android Cupcake, Donut, GingerBread, HoneyComb, Ice Cream Sandwich
		, dan versi terbarunya untuk saat ini adalah jelly bean. Versi android ini mendukung beberapa aplikasi seperti google now, google assistant, notifications, dan screen capture.
		Disetiap versinya android dilengkapi dengan API yang bertujuan untuk mengidentifikasi aplication programming interface.
	
	\subsection(Contoh Fitur-Fitur dalam Android)
		Di dalam Android terdapat fitur-fitur penting yang wajib anda ketahui pada bagian bawaan OSnya yaitu:
		1.	Android memiliki Fitur GPS yang mencari lokasi terdekat untuk mencari keberadaan anda saat ini berdasarkan referensi \cite{anwar2014implementasi}
		2.	Android memiliki Fitur Menguatkan Sinyal saat kondisi tidak menentu.
		3.	Android memiliki Aplikasi Dukungan dari PlayStore untuk mengunduh instalasi aplikasi gratis pada smartphone
		4.	Android memiliki Daya Tahan Baterai yang cukup dan bisa bertahan dengan kondisi smartphone tidak menggunakan paket data internet
			hingga 2 hari maksimalnya.
		5.	Android memiliki aplikasi penyimpanan data yang luas untuk menyimpan data pribadi anda. Tetapi ini sangat bergantung pada spesifikasi
			Smartphone anda yang pakai saat ini. Kapasitas data saat peluncuran pertama menyediakan simpanan sekitar 1 GB, Seiring waktu berjalan
			Penyimpanan data semakin di perluas pada smartphone android hingga 32gb sampai sekarang.
		6. 	Android memiliki fitur sistem penyeimbangan hardware yang diluncurkan untuk mengoptimasikan performa smartphone untuk menghindari terjadinya
			kesalahan teknis atau istilahnya sebagai bug dalam menjalankan sistem Android. Biasanya optimasi smartphone ini dijalankan saat aplikasi digunakan
			dijalankan secara berlebihan. Contohnya bermain Mobile Legends atau Garena AOV secara tiba-tiba mengalami lag atau bug saat aplikasi berlangsung.
		7. 	Android memiliki aplikasi alarm sebagai pengganti jam dinding anda untuk membangunkan tidur anda yang terlelap. Banyak keunikan aplikasi ini,
			Anda bisa mengatur suara musik sesuai selera teman-teman semua. Selain itu bisa mengatur volume suara yang akan diujikan saat alarm berbunyi seberapa nyaringnya suara akan terdengar
		8.	Android memiliki fitur backup data yang digunakan untuk menyimpan data penting anda di server awan atau Cloud Server apabila data-data smartphonemu tidak sengaja terhapus aplikasi yang sudah diinstal sebelumnya.
			Tidak perlu khawatir tentang kehilangan data anda. Selama smartphone anda di sinkronasi secara menyeluruh, Semua data akan tersimpan dan dapat di sinkronasikan pada pengguna smartphone yang lain.
		9.	Android memiliki fitur Launcher untuk menunjukkan semua aplikasi bawaan android yang terinstal pada smartphone anda.
		10.	Android memiliki aplikasi Backup dan Restore. Berbeda dengan Cloud Server, aplikasi ini diluncurkan untuk menyimpan data anda keseluruhan pada 1 tempat tertentu baik itu cloud server ataupun lewat sd card.
			untuk disimpan sewaktu-waktu anda ingin menggantikan smartphone lama anda kepada orang lain apabila semua mau disimpan sesuai keperluan masing-masing pengguna smartphone.
		11.	Android memiliki aplikasi buku untuk dibaca pada smartphone dan dapat menggantikan buku yang berupa isi kertas dan pencetakan. Aplikasi ini sangatlah fleksibel karena bisa dibawa kemana saja tanpa perlu membawa-bawa
			buku dalam jumlah banyak. Diperlukannya sebuah SD Card untuk menyimpan buku anda di smartphone android anda.
		12.	Android memiliki aplikasi kalkulator yang menyeluruh untuk menghitung jumlah angka yang tak terhingga dengan batasan beberapa digit. Biasanya batasan digit yang dibuat oleh android sebanyak 9 angka digit
			untuk menghindari jumlah numerik tak terhingga karena kerja sistem android yang terbatas.
	
	\section(Kelebihan dan Kekurangan OS Android)
		OS Android ini memang bagus dari semua segala aspek, Tetapi banyak sekali yang harus kita rangkul bahwa android mempunyai dampak yang mempengaruhi penggunaan yang harus diperhatikan. Karena android pada umumnya masih banyak revisi
		yang harus diperbaiki dalam dukungan OS-Nya di seluruh smartphone untuk lebih kompatibel digunakan dan sesuai aturan pakai. Berikut Kelebihan dan Kekurangan dari OS Android.
	
	\subsection(Kelebihan OS Android)
		Inilah beberapa manfaat kelebihan pada penggunaan OS Android yaitu, sebagai berikut :
		1.	Android ini mempunyai dasar algoritma pemrograman menggunakan C+ dan Java. Manfaat yang diberikan memberikan struktur yang simpel dan mudah dalam
			koding bahasa pemrograman OS Android.
		2. 	Android didesain untuk penggunaan yang lebih cepat dan responsif pada berbagai aplikasi yang terinstal dan digunakan. Baik Multimedia, Games, dan sebagainya.
		3.	Android dirilis di Pasar dengan harga relatif murah dari segi OS-Nya. Karena Android ini dirancang untuk menghindari kepalsuan saat pemasaran smartphone android di masyarakat.
			Bertujuan untuk menghindari terjadinya pemalsuan yang mengaku sebagai developer pengembang Android.
		4.	Android siap menghadirkan fitur-fitur aplikasi terbaru dan bertanggung jawab dalam riset perkembangan OS Android di Masa Depan.
		5.	Android berkomitmen memberikan layanan terbaik kepada masyarakat apabila ada permasalahan yang terjadi saat menggunakan OS Android tersebut.
		6.	Android sudah banyak merilis versi android sesuai dengan ponsel dan smartphone anda. Tidak semua ponsel diupgrade karena dari segi keamanan.
		7.	Android membuat keringanan pada pembuatan smartphone karena fitur layar sentuh memberikan tampilan keyboard apabila ingin mengetik sebuah kata.
			Sehingga dalam pembuatan smartphone pada waktu kurun lama memberikan rilisan terbaru smartphone yang lebih canggih lagi.
		8.	Android ini siap diberikan cap label dengan hasil pendapat masyarakat sebagai OS yang sangat positif dari segi penggunaannnya. Membuat Android ini semakin diminati semua orang.
			
		Dan masih banyak Kelebihan OS Android lainnya yang disebutkan diatas secara meluas. Bagaimanapun juga Kelebihan ini memberikan dampak yang baik pada era teknologi yang semakin maju ini.
		Walaupun kelebihan ini tidak secara menyeluruh android berkomitmen untuk membuat hati pelanggan puas dalam mendukung penggunaaan OS Android ini menjadi lebih baik
		di masa yang akan datang.
		
	\subsection(Kekurangan OS Android)
		Mungkin anda belum sempat berpikir bahwa masih banyak kekurangan pada permasalahan yang dihadapi pada OS Android ini. Tetapi developer Android selalu mengambil langkah lebih maju untuk mengurangi
		kekurangan pada permasalahan di OS Android. Berikut beberapa kekurangan pada penggunaan OS Android.
		1.	Android ini bisa dibongkar alias jika anda mempunyai OS Android versi No Root(Versi Original) bisa dilakukan Rooting Device atau dikenal sebagai Pembajakan OS Android sama halnya dengan crack key OS Windows.
			Tetapi Root disini bertujuan untuk melakukan sistem hacking pada aplikasi tertentu untuk mencari keuntungan sendiri yang bersifat merugikan android dalam rilisan OS Android ini.
			Root device ini biasanya menginstall aplikasi illegal yang biasanya aplikasi sebelumnya tidak bisa terinstal karena alasan tertentu sehingga oknum nakal pengguna smartphone melakukan tindakan kejahatan
			pada OS Android.
		2.	Android memiliki banyak bug atau aplikasi rusak yang belum siap diperbaiki dalam kurun waktu lama. Karena masih banyak sifat smartphone belum mendukung dalam penggunaan OS ini. Meskipun smartphone sudah berbasis versi android terbaru
			banyak hal yang perlu dibenahi bahwa pendukung OS Android blum tentu penuh sempurna mendukung smartphone baru sehingga banyak rilisan android versi terbaru untuk menutupi kasus bug atau laporan kerusakan
			karena kurang dukungan aplikasi yang kompatibel pada smartphone.
		3.	Android ini banyak digunakan para hacker atau pembajak untuk melakukan aksi kriminal berupa penipuan dengan menunjukkan link website, telpon 
			tak dikenal. Biasanya ada aplikasi tersembunyi android yang tidak boleh dipakai karena sifat ini sangatlah
			menentang aturan hukum dan pasal yang bersangkutan penggunaan smartphone.
		4.	Android tidak bisa diupdate ke versi terbaru karena masalah tidak mendukungnya sarana prasana smartphone dari segi aplikasinya. Tidak semua diupdate karena banyak aplikasi tidak mendukung saat melakukan update versi android ke versi terbarunya.
			Saat banyak orang berpikir android memiliki kinerja baik tapi untuk segi smartphone atau ponsel masih belum dikembangkan lebih lanjut karena belum memiliki alasan tertentu.
			Dan mungkin saat ini kekurangan android yang banyak terjadi di masyarakat. Walau ada kekurangan kita sebagai penggunanya harus memberikan dukungan para pengembang android untuk menjadi lebih baik di masa depan.
	
	\section(Contoh logo Android)
		Ini adalah sebuah gambar logo Android \ref{androidfigures}
		Logo ini dibuat sendiri tanpa mengambil dari Hak Cipta orang lain.
		Hak Cipta Gambar ini dibuat oleh Yusuf Al-Qardhawi dan dibuat menggunakan Adobe Photoshop CS5
		
	\section(Kesimpulan)
		Android \ref{androidfigures} memiliki banyak inovasi dalam prospek pengembangan sistem operasinya untuk menjadi lebih baik
		di masa depan. Karena tidaklah mudah membuat sesuatu yang berhasil tanpa usaha keras. Sebagai Mahasiswa
		dan Mahasiswi untuk mendukung penemu pengembangan Android ini karena tanpa mereka smartphone atau ponsel
		pada saat ini belum mengalami perubahan secara pesat.